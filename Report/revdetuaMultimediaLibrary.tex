\documentclass[a4paper]{revdetua}

\usepackage[english]{babel}
\usepackage{cite}
\usepackage{ifthen,graphicx,color}
\usepackage[latin1]{inputenc}
\usepackage{url}

\oddsidemargin  -0.5cm
\evensidemargin -0.5cm
\textwidth      17cm
\topmargin      -2.5cm
\headheight     0cm
\textheight     26.5cm

\begin{document}

\title{Multimedia Library: Use of SIMD instructions to speedup the processing of vectors and matrixes}

\author{Edgar Domingues, M�rio Antunes, Nuno Lau}

\maketitle

\begin{abstract}
This document presents a study on how use SIMD(Single Instruction Multiple Data) instructions to speedup 
the processing of vectors and matrixes, particulary instructions of SSE(Streaming SIMD Extensions) family.
For this study was developed a library with classic version of the algorithm and a version using SSE 
instructions for the most common opera��es in vectors and matrixes.
\end{abstract}

\begin{resumo}
Este documento apresenta um estudo sobre como utilizar instru��es SIMD (Single Instruction Multiple Data)
para acelerar o processamento de vectores e matrizes, em part�cular instru��es da fam�lia SSE(Streaming
SIMD Extensions). Para este estudo foi desenvolvida uma biblioteca que possui as vers�es cl�ssicas e as vers�es
que utilizam instru��es multim�dia das opera��es mais comuns de vectores e matrizes.
\end{resumo}

\begin{keywords}
SIMD, SSE
\end{keywords}

\begin{palavraschave}
SIMD, SSE
\end{palavraschave}

\section{Introduction}

The need to improve the performance o computing system leads to the approach to
parallelize the software. A technique used is SIMD(REF), executes the same
operation on multiple data simultaneously. This is an advantage in multimedia
system, where is performed the same operation over a wide number of data, for
instance, manipulation image programs, where the same operation is performed for
all, or a wide range, of pixels from a image.

The MMX and SSE are sets of instructions SIMD developed by Intel and have been
expanded ever since.

For this study we decided only use SSE, because it have a wider instrcution set,
that allows to explore more complex algorithms. Allows to process faster a set
of data because uses register width the twice of the size, witch means, that the
algorithms will have a greater performance.

The program language choosed for this library was C++, because is a well known, 
well documented and allows to insert ASM code, this is particulary important
because SSE is a set of low level instructions and are avaiable only from ASM code.
\section{Multimedia Extensions for processors}

Most of the actual processor supports a set of SIMD instructions in their ISAs
(Instructions Set Architectures). The first ISA to introduce multimedia
extensions was PA-RISC, in January 1994 they introduce MAX-1(Multimedia 
Acceleration eXtensions). Later on Sun(or Oracle??) added VIS(Visual Instruction
Set) to the Sparc ISA follow by HP that introduced MAX-2, its second generation
multimedia extensions for its 64-bit PA-RISC 2.0 ISA.

In January 1997 Intel introduced chips with MMX (Multi-Media Extensions) added
to the ix86 ISA.

In 1999 Intel presents SSE with its processor Pentium III, since then SSE
extanded with every version, and its support by almost every desktop processor.
The latest version is 4.2, and can be found in the Core processor from Intel. A
new version is being developed by AMD, SSE5.

For this study we only used SSE and SSE2, because the new versions only add very
specific instruction to a very specific operation, and the porpuse of the study
is a generic library for vector and matrixes processing.
\bibliography{Refs}{...}

\end{document}
